\documentclass[10pt]{article}
\usepackage[margin=0.9in]{geometry}
\usepackage{amsmath,amssymb,mathtools}
\usepackage{graphicx}
\usepackage{hyperref}
\setlength{\parskip}{4pt}
\setlength{\parindent}{0pt}

\begin{document}
\begin{center}
{\Large FRC 100.004 — Quantum Foundations in Fractal Resonance Cognition}\\
{\large October 2025}\\[4pt]
H. Servat
\end{center}

\section*{Abstract}
We propose a deterministic, resonance--based account of measurement and entanglement. The central idea is that ``collapse'' corresponds to phase--locking (attractor selection) in a coherence field, so that outcomes appear random macroscopically while the dynamics is lawful. We present a minimal extension of standard open--system dynamics with a small coherence drift, derive qualitative predictions (weak pre--collapse drift; dephasing asymmetry near pointer coupling), and provide reproducible simulations that contrast Fractal Resonance Cognition (FRC) with baseline quantum models. The program is falsifiable: if the predicted pre--collapse signatures are absent in weak--measurement protocols, the resonance hypothesis is ruled out in the tested regime.

\section*{1. Introduction}
Mainstream interpretations (Copenhagen, Many--Worlds, Bohm, GRW) disagree on the nature of collapse. FRC posits a simpler mechanism: measurement outcomes are the result of a resonance process that phase--locks the system--apparatus state to a pointer attractor. This paper formalizes that idea in the smallest possible way and lists concrete experimental discriminants.

\section*{2. Minimal Formalism}
Let $\rho(t)$ be the density operator. Write a baseline open--system equation $\dot\rho=L[\rho]$ (e.g., Lindblad), and introduce a small coherence drift
\begin{equation}
\dot\rho \;=\; L[\rho]\; +\; \alpha\,\nabla_{\rho}\,\ln C[\rho], \qquad 0<\alpha\ll 1,
\end{equation}
with a coherence functional $C[\rho]$ (we use $C[\rho]=\exp[-S(\rho)/k_*]$; $S$ may be von Neumann or a tractable proxy). In the limit $\alpha\to 0$ we recover standard QM. The drift encodes a tendency to ascend the coherence gradient; Appendix~A sketches a dissipation inequality inherited from the FRC~566 UCC.

\paragraph{Measurement model (pointer basis).}
Couple a system observable $A$ to an apparatus pointer via $H_P=g\,A\otimes P$. For $g>0$, the pointer basis of $A$ becomes a resonant attractor family. The coherence drift weakly biases trajectories toward these attractors; Born weights are recovered as $\alpha\to 0$.

\section*{3. Predictions (falsifiable)}
\textbf{(P1) Weak pre--collapse drift.} In sequential weak measurements prior to a strong readout, the mean coherence exhibits a small ascent $\Delta S\approx -k_*\Delta\ln C>0$ before phase--lock.\newline
\textbf{(P2) Dephasing asymmetry vs pointer coupling.} In interferometers with tunable pointer coupling $g$, FRC predicts a small, systematic deviation in visibility $\mathcal{V}(g)$ relative to standard open--system fits; curves separate near resonant match.

\section*{4. Simulations (reproducible)}
We provide two minimal simulations (code/100.004/): (i) a weak--measurement toy showing pre--collapse drift and locking time distributions vs $\alpha$; (ii) an interferometer visibility toy comparing standard vs FRC curves as a function of coupling $g$. Figures are generated by \texttt{make\_figures.py} and written to \texttt{artifacts/100.004/}.

\begin{center}
\includegraphics[width=0.75\linewidth]{../../artifacts/100.004/weak_drift.png}\\
\emph{Figure 1.} Weak pre--collapse drift and locking times vs $\alpha$ (toy model; seeds fixed).
\end{center}

\begin{center}
\includegraphics[width=0.75\linewidth]{../../artifacts/100.004/visibility_vs_g.png}\\
\emph{Figure 2.} Interferometer visibility $\mathcal{V}(g)$: standard vs FRC toy fits (separation near resonant match).
\end{center}

\section*{5. Comparisons and Limits}
\begin{itemize}
  \item \textbf{Copenhagen:} collapse postulated; no dynamical mechanism.
  \item \textbf{Many--Worlds:} unitary only; effective collapse by branching; FRC posits real phase--locking with small drift.
  \item \textbf{Bohm:} deterministic trajectories; FRC is deterministic in coherence space rather than position.
  \item \textbf{GRW:} stochastic collapse; FRC uses deterministic drift with noise only through the environment.
\end{itemize}
Limits: small--$\alpha$ regime; energy accounting; no--signaling constraints; falsifiability via (P1) and (P2).

\section*{6. Reproducibility}
Code is under \verb|code/100.004/| with a one--command script \verb|make_figures.py|. Figures are regenerated into \verb|artifacts/100.004/|; random seeds are fixed for exact reproduction.

\section*{References}
\small
\begin{itemize}
  \item FRC~566.001 — Entropy--Coherence Reciprocity and UCC. DOI: \href{https://doi.org/10.5281/zenodo.17437759}{10.5281/zenodo.17437759}.
  \item FRC~100.003 — Resonant Collapse: Guided Wavefunction Collapse via Resonant Attractors. DOI: \href{https://doi.org/10.5281/zenodo.15079820}{10.5281/zenodo.15079820}.
  \item FRC~100.003.566 — UCC and Dissipation (Scientific Note). DOI: \href{https://doi.org/10.5281/zenodo.17437878}{10.5281/zenodo.17437878}.
\end{itemize}

\section*{Appendix A: Dissipation Sketch}
Let $C=\exp[-S/k_*]$. In the open--system setting with drift, one obtains a nonnegative production term $\propto \int\!\|\nabla\ln C\|^2$ under standard boundary conditions, consistent with the UCC dissipation in FRC~566.

\end{document}

