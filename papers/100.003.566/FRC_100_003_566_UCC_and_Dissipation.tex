\documentclass[10pt]{article}
\usepackage[margin=0.9in]{geometry}
\usepackage{amsmath,amssymb,mathtools}
\usepackage{graphicx}
\usepackage{hyperref}
\setlength{\parskip}{4pt}
\setlength{\parindent}{0pt}

\begin{document}
\begin{center}
{\Large FRC 100.003.566 — UCC and Dissipation (Scientific Note)}\\
{\large October 2025}\\[4pt]
H. Servat
\end{center}

\section*{Abstract}
We present a concise, scientific statement of the Universal Coherence Condition (UCC)
\begin{equation}
 \partial_t \ln C = -\nabla\!\cdot J_C + S_C,\qquad J_C = -D_C\,\nabla \ln C,\; D_C>0,
\end{equation}
and the resulting dissipation inequality
\begin{equation}
 \sigma(t) = k_* D_C \int \|\nabla\ln C\|^2\, dV \;\ge\; 0
\end{equation}
under standard boundary conditions. We include a minimal 1D diffusion demonstration with Neumann boundaries showing monotone decay of $\int\|\nabla\ln C\|^2$. This note is a $\mu$-free, unit–consistent companion to FRC 566.001 and cited by 567.901.

\section*{1. UCC and Boundary Conditions}
For a dimensionless coherence $C>0$, $\ln C$ is well defined. With $J_C=-D_C\nabla\ln C$ and $D_C>0$, the UCC takes the diffusion–reaction form above. Under Neumann or Dirichlet boundary conditions, multiplying by $\ln C$ and integrating by parts yields a nonnegative production term proportional to $\int\|\nabla\ln C\|^2$ (details omitted for brevity; see 566.001 for the reciprocity context and units).

\section*{2. Numerical Demonstration (1D)}
We evolve an initial hump in $\ln C$ on $[0,1]$ with Neumann boundaries and $D_C=0.05$ using a centered finite–difference scheme. The energy–like quantity $\int\|\nabla\ln C\|^2$ decays monotonically, consistent with the dissipation inequality.

\begin{center}
\includegraphics[width=0.7\linewidth]{../../artifacts/566/ucc_dissipation.png}\\
\emph{Figure.} UCC dissipation demonstration with Neumann BCs.
\end{center}

\section*{3. Reproducibility}
Code: \verb|code/566/frc_566_ucc_sim.py|. Run it from the project root to regenerate the figure:
\begin{verbatim}
python code/566/frc_566_ucc_sim.py
\end{verbatim}
We recommend $k_*=1$ for information–layer experiments and $k_*=k_B$ for thermo–physical projections.

\end{document}
